\documentclass[a4paper,11pt]{article}
\usepackage[utf8]{inputenc}
\usepackage[russian]{babel}
\usepackage[T1]{fontenc}
\usepackage{hyperref}

\author{Александр Полозов, Олег Смирнов\\
\texttt{polozov.alex@gmail.com}, \texttt{oleg.smirnov@gmail.com}}
\date{\today}
\title{Алгоритмы: построение и анализ -- программа курса}

\begin{document}
\maketitle
\begin{abstract}
Программа базируется на курсе ``6.046: Introduction to Algorithms'' 
Массачусетского технологического института\footnote{\href{http://goo.gl/jIOiq}
{http://ocw.mit.edu/ ... /6-046j-introduction-to-algorithms-sma-5503-fall-2005}},
на одноимённом учебнике Т. Кормена, Ч Лейзерсона, Р. Ривеста и К. Штайна,
а также учебниках ``Фундаментальные алгоритмы на C++'' Р. Седжвика и
``Программирование: теоремы и задачи'' А. Шеня.
\end{abstract}
\section*{I семестр}
\subsection*{Модуль 1. Анализ алгоритмов}
\begin{itemize}
\item Тема 1.1. Обзор курса. Анализ алгоритмов. Сортировка вставкой
(Insertion sort) и сортировка слиянием (Merge sort). Бинарный поиск
\item Тема 1.2. Асимптотическая нотация. Анализ рекуррентные соотношений.
Основная теорема
\item Тема 1.3. Парадигма ``Разделяй и властвуй''. Быстрое возведение в степень.
Числа Фибоначчи. Алгоритм Евклида
\item Тема 1.4. Стеки и очереди. Пирамидальная сортировка (Heapsort).
Очереди с приоритетами
\item Тема 1.5. Сортировка Quicksort. Рандомизированные алгоритмы
\item Тема 1.6. Сортировка за линейное время. Сортировка подсчётом
(Counting sort). Поразрядная сортировка (Radix sort)
\item Тема 1.7. Хэширование и хэш-функции. Идеальное хэширование
\end{itemize}

\section*{Модуль 2. Построение алгоритмов}
\begin{itemize}
\item Тема 2.1. Динамическое программирование I. Задача поиска наибольшей
общей подстроки (LCS), задача о кратчайшем двунаправленном пути, задача о
наибольшей возрастающей подпоследовательности (LIS), задача о выдаче сдачи
\item Тема 2.2. Динамическое программирование II. Задача о ранце, задача о
покрытии шахматной доски
\item Тема 2.3. Двоичные поисковые деревья. Обход дерева и варианты записи
дерева. Связь с Quicksort
\item Тема 2.4. Амортизационный анализ. Реализация динамических таблиц
(Vector) и cистемы непересекающихся множеств (Disjoint-set)
\item Тема 2.5. Графы. Варианты записи графа. Поиск циклов. Поиск в ширину
(BFS), в глубину (DFS) и топологическая сортировка. Связность графа. Двудольный
граф
\item Тема 2.6. Жадные алгоритмы. Задача минимального остова (MST). Алгоритм
Прайма. Алгоритм Краскала
\item Тема 2.7. Кратчайшие пути I. Свойства и применение. Алгоритм Дейкстры
\item Тема 2.8. Кратчайшие пути II. Алгоритм Беллмана-Форда. Задача разностных
ограничений
\item Тема 2.9. Кратчайшие пути III. Все пары вершин. Алгоритм Флойда-Уоршолла.
Алгоритм Джонсона. Транзитивное замыкание
\end{itemize}

\section*{II семестр}
\subsection*{Модуль 3. Дополнительные темы}
\begin{itemize}
\item Тема 3.1. Задача минимума на отрезке (RMQ) и решение корневой
декомпозицией. Дерево отрезков
\item Тема 3.1. Offline и online версии задачи RMQ. Задача SRMQ
\item Тема 3.2. Задача наименьшего общего предка (LCA). Сведение LCA к RMQ
\item Тема 3.3. Сбалансированные деревья. Пример реализации вставки в AVL
\item Тема 3.4. Декартово дерево (Treap)
\item Тема 3.5. Порядковые статистики. Динамические порядковые статистики
\item Тема 3.6. Конкурентый анализ на примере самоупорядочивающихся списков
\item Тема 3.7. Вычислительная геометрия I. Векторное представление. Выпуклая 
оболочка
\item Тема 3.8. Вычислительная геометрия II. Точка в полигоне. Площадь 
полигона. Метод заметания.
\item Тема 3.9. Строковые алгоритмы I. Префикс-функция. Z-функция. Хэширование
\item Тема 3.10. Строковые алгоритмы II. Бор. Алгоритм Рабина-Карпа
\item Тема 3.11. Максимальные поток и минимальный разрез. Алгоритм 
Форда-Фалкерсона. Паросочетания
\end{itemize}
\end{document}
