\documentclass[a4paper,11pt]{article}
\usepackage[utf8]{inputenc}
\usepackage[russian]{babel}
\usepackage[T1]{fontenc}
\usepackage{amssymb,amsmath,indentfirst}

\author{Олег Смирнов\\
\texttt{oleg.smirnov@gmail.com}}
\date{\today}
\title{Алгоритмы: построение и анализ -- задачи}

\begin{document}
\section*{Раздел 1. Темы 1.1-2}
\textbf{Упражнение 1-1} В сортировке вставкой используется линейный поиск для просмотра (в обратном порядке) отсортированного подмассива $A[1 \ldotp\ldotp j-1]$. Можно ли использовать бинарный поиск вместо линейного, чтобы время работы этого алгоритма стало $O(n \lg n)$?

\textbf{Упражнение 1-2} Доказать, что время работы алгоритма равно $\Theta(g(n))$ тогда и только тогда, когда время работы алгоритма в наихудшем случае равно $O(g(n))$, а время работы в наилучшем случае равно $\Omega(g(n))$.

\textbf{Упражнение 1-3} Найти точную верхнюю и нижнюю асимпототическую оценку для каждой из рекуррентностей. Считать $T(n)$ константным для $n \leqslant 10$.
\begin{itemize}
\item $T(n) = 2 T(n/3) + n \lg{n}$
\item $T(n) = 3 T(n/5) + \lg^2 {n}$
\item $T(n) = T(n/2) + 2^n$
\item $T(n) = T(\sqrt{n}) + \Theta(\lg \lg n)$
\item $T(n) = 10 T(n/3) + 17 n^{1.2}$
\item $T(n) = 7 T(n/2) + n^3$
\item $T(n) = T(n/2 + \sqrt{n}) + \sqrt{6046}$
\item $T(n) = T(n-2) + \lg{n}$
\item $T(n) = T(n/5)+ T(4n/5) +\Theta(n)$
\item $T(n) = \sqrt{n} T(\sqrt{n}) + 100 n$
\end{itemize}

\textbf{Задача 1-1} Дано $n$ отрезков $[a[i]; b[i]]$ на прямой $(i = 1 \ldotp\ldotp n)$. Найти максимальное $k$, для которого существует точка прямой, покрытая $k$ отрезками (``максимальное число слоёв''). Число действий -- порядка $n \lg n$.

\textbf{Задача 1-2} Массив $A[1 \ldotp\ldotp n]$ называется унимодальным, если содержит возрастающую последовательность, за которой идёт убывающая последовательность. Т.е. существует индекс $m \in \{1, 2, \ldotp\ldotp n\}$ такой, что
\begin{itemize}
\item $A[i] < A[i+1]$ для всех $1 \leqslant i < m$, и
\item $A[i] > A[i+1]$ для всех $m \leqslant i < n$
\end{itemize}
Здесь $A[m]$ -- локальный максимум, окружённый меньшими элементами ($A[m-1]$ и $A[m+1]$). Необходимо найти максимальный элемент унимодального массива $A[1 \ldotp\ldotp n]$ за $\lg n$ операций.

\textbf{Задача 1-3} Дано $n$ точек на плоскости. Построить их выпуклую оболочку -- минимальную выпуклую фигуру, их содержащую. (Резиновое колечко, натянутое на вбитые в доску гвозди -- их выпуклая оболочка.) Число операций не более $n \lg n$.

\section*{Раздел 1. Темы 1.3-4}

\textbf{Упражнение 1-} Придумать способ моделирования очереди с помощью двух стеков (и фиксированного числа переменных типа T). При этом отработка $n$ операций с очередью (начатых, когда очередь была пуста) должна требовать порядка $n$ действий.

\textbf{Упражнение 1-} Деком называют структуру, сочетающую очередь и стек: класть и забирать элементы можно с обоих концов. Как реализовать дек ограниченного размера на базе массива так, чтобы каждая операция требовала константного числа действий?

\end{document}
